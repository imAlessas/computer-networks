\title{\vspace{160px} \textbf{\huge{Multimedia}} \\\vspace{17.5px} \LARGE{Homework 1}  \vspace{10px}}
\author{Alessandro Trigolo}
\date{30 Aprile 2024}

\begin{document}

\maketitle\newpage




\section{Obiettivo}
\todo{Descrivi decentemente l'obiettivo}



\section{Codice sorgente}
\todo{Fai introduzione}

\subsection*{Task 1}

La prima richiesta dell'homewrok è quella di selezionare un'immagine e calcolarne l'entropia. 

\begin{lstlisting}
    img_file_name = 'rocky-beach-greys.jpg';
    img_folder = '../imgs/';
    
    img = imread("" + img_folder + img_file_name);
    
    % creates figure and settings
    f = figure(1);
    f.Name = 'Image';
    f.NumberTitle = 'off';
    f.Position = [450, 100, 700, 600];
    
    imagesc(img); colormap(gray); axis image; axis off; 
\end{lstlisting}




\section{Conclusioni}

\end{document}
