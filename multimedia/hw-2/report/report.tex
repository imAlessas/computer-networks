\title{\vspace{160px} \textbf{\huge{Multimedia}} \\\vspace{17.5px} \LARGE{Homework 2}  \vspace{10px}}
\author{\href{https://github.com/imAlessas}{Alessandro Trigolo}}
\date{7 Giugno 2024}

\begin{document}

\maketitle\newpage

\tableofcontents
\vspace{50px}
\listoffigures
\newpage

\section{Introduzione}
\todo{Fai introduzione}

\vspace{15px}\subsection{Utilizzo dello script}
Assolutamente necessario avere il sistema operativo in italiano altrimenti lo script si schianta

\vspace{15px}\subsection{Parametri di progetto}

\begin{itemize}
    \item Los Angeles
    \item K = 100
    \item Dim = 32 (Default)
\end{itemize}

\vspace{35px}\section{Numero di link attraversati}

Con LA, si hanno 12 links sia con tracert che con ping. A volte con il ping si ha 13 anziche 14 (valuta se inserirlo)

Task 1
 Number of links found with tracert: 12
                47.08 sec
 Number of links found with muliple ping: 12
                44.3 sec

\vspace{35px}\section{Analisi del \textsl{Round Trip Time}}

\todo{Inserisci qui tutte le immagini e analizzale}


\vspace{35px}\section{Throughput}

Devi rifare tutto perchè non te li sei salvati, a sto punto fai anche 200 istanze pimpanti e lo fai runnare 20 minuti

\vspace{35px}\section{Conclusioni}



\end{document}
